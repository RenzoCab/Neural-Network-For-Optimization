\documentclass[12pt]{article}
\usepackage[table]{xcolor}
\usepackage[margin=1in]{geometry} 
\usepackage{amsmath,amsthm,amssymb}
\usepackage[english]{babel}
\usepackage{tcolorbox}
\usepackage{enumitem}
\usepackage{hyperref}
\usepackage{listings}
\usepackage{blkarray}
\usepackage{float}
\usepackage{bbm}
\usepackage{bm}
\usepackage{subfigure}
\usepackage{commath}
\usepackage{booktabs}
\usepackage{siunitx}
\usepackage{tcolorbox}
\usepackage{mathtools}
\usepackage[usestackEOL]{stackengine}

\setcounter{secnumdepth}{5}
\setcounter{tocdepth}{5}

\newtheorem{theorem}{Theorem}[section]
\newtheorem{corollary}{Corollary}[theorem]
\newtheorem{lemma}[theorem]{Lemma}
\newtheorem{proposition}[theorem]{proposition}
\newtheorem{exmp}{Example}[section]\newtheorem{definition}{Definition}[section]
\newtheorem{remark}{Remark}
\newtheorem{ex}{Exercise}
\theoremstyle{definition}
\theoremstyle{remark}
\bibliographystyle{elsarticle-num}

\DeclareMathOperator{\sinc}{sinc}
\newcommand{\RNum}[1]{\uppercase\expandafter{\romannumeral #1\relax}}
\newcommand{\N}{\mathbb{N}}
\newcommand{\Z}{\mathbb{Z}}
\newcommand{\R}{\mathbb{R}}
\newcommand{\E}{\mathbb{E}}
\newcommand{\matindex}[1]{\mbox{\scriptsize#1}}
\newcommand{\V}{\mathbb{V}}
\newcommand{\Q}{\mathbb{Q}}
\newcommand{\K}{\mathbb{K}}
\newcommand{\C}{\mathbb{C}}
\newcommand{\prob}{\mathbb{P}}

\lstset{numbers=left, numberstyle=\tiny, stepnumber=1, numbersep=5pt}

\begin{document}
\title{PROJECT PROPOSAL	:\\
Optimization using Neural Networks}
%\author{Renzo Caballero}
\date{}
\maketitle

\section*{Original formulation:}

In this project, we want to approximate the result of a minimization problem using a neural network (NN). Our problem $P(\cdot)$ is composed of a linear objective with linear and box constraints:
\begin{equation}
P(\bm{b,d},c)=\left[\min_{\bm{\phi}}\left[\bm{\phi}^T\bm{d}\right]\quad\text{subject to}\quad\begin{cases}
0\leq\bm{\phi}\leq1\\
\bm{\phi}^T\bm{b}=c
\end{cases}\right],
\label{1}
\end{equation}
where $\bm{\phi,b,d}\in\R^5$, $c\in\R$, and then, $P:\R^{11}\to\R$. We define the subset $\mathcal{A}\subset\R^{11}$ as the set where the values of $\bm{b,d}$, and $c$ can exists.\\
\quad\\
We want to find an approximate function $\tilde{P}(\cdot)$ such that
\begin{equation*}
||P-\tilde{P}||_{L^{\infty}(\mathcal{A})}\leq C,
\end{equation*}
where $C$ is an arbitrary positive constant, the idea is that evaluate $\tilde{P}(\cdot)$ is faster than evaluate the original function $P(\cdot)$.

\subsubsection*{Suggested approximation}

The function $\tilde{P}(\cdot)$ can be a trained NN using real solutions of (\ref{1}).

\subsubsection*{Suggested literature and tutorials}

\begin{enumerate}

\item \url{https://scikit-learn.org/stable/}. {\color{blue} It has tutorials, and it is suitable for standard NN.}

\item \url{https://pytorch.org/}. {\color{blue} It is useful if you want to design your architecture or to play with loss functions.}

\item \url{https://www.fast.ai/}. {\color{blue} It is more practical and friendly for people no familiar with math.}

\end{enumerate}

These links were suggested by a Ph.D. student researching AI during an AI summer school (\url{http://acai2019.tuc.gr/}).

\section*{Second formulation (Lanza):}

In this second formulation we want to approximate
\begin{equation}
P(\bm{b,d,Q},c)=\left[\min_{\bm{\phi}}\left[\bm{\phi}^T\bm{d}\right]\quad\text{subject to}\quad\begin{cases}
0\leq\bm{\phi}\leq1\\
\bm{\phi}^T\bm{Q}\bm{\phi}+\bm{\phi}^T\bm{b}=c
\end{cases}\right],
\label{2}
\end{equation}
where $\bm{\phi,d,b}\in\R^{12}$, and $Q\in M^{12\times12}$. As the matrix $\bm{Q}$ has only 8 non-zero entries, we storage it as a vector.

\subsection*{About the data}

The data is in Matlab format, and they are cells. Each file has a cell with dimensions $1\times225$. Each one of these cells has inside a $1\times5$ cell with 5 arrays. The arrays have dimensions 12, 1, 12, 8, and 1, respectively. These 5 arrays represent a single data point, where the first 4 arrays are the input, and the 5-th array is the output.\\
\quad\\
In other words, each data point is composed by:
\begin{equation*}
\left(\underbrace{\bm{b},c,\bm{d},\bm{Q}}_{Inputs},\underbrace{P(\bm{b,d,Q},c)}_{Output}\right).
\end{equation*}

\end{document}